% !TEX root = ../document.tex
\section{Fazit}
\label{sec:Fazit}
Zusammenfassend lässt sich sagen, dass Blockchain als Datenbank für eine persistente, manipulationsfreie Datenspeicherung sinnvoll ist. Mit der Software Multichain besteht die Möglichkeit mit wenigen Schritten eine funktionierende Blockchain aufzusetzen.

Das Interesse an einer produktiven Nutzung im unternehmerischen Umfeld wird weiter ansteigen. Der Aspekt der Integrität ist hierfür ausschlaggebend. Im Bereich der Blockchain als Datenbank stehen aber noch viele Fragen aus. Beispielsweise besteht theoretisch die Möglichkeit bei einer Kontrolle von mehr als 50 Prozent der Nodes, die Blockchain zu manipulieren. Des Weiteren spielen auch datenschutzrechtliche Fragen eine Rolle. Beispielsweise lassen sich Blockchains nicht rückwirkend verändern, demnach können keine sensitiven oder illegalen Daten gelöscht werden. In Zukunft müssen weitere Studien über die produktiven Ansätze der Blockchain-Technologie durchgeführt werden.