% !TEX root = ../document.tex
\section{Einleitung}
\label{sec:Einleitung}
Seit der Sch{\"o}pfung der ersten Bitcoins 2009 steigt die Popularit{\"a}t der Kryptow{\"a}hrungen und der Blockchain-Technologie stetig an.\footnote{\cite[S.~312]{Neugebauer.2018}} Besonders die wirtschaftliche Institutionen besch{\"a}ftigen sich intensiver mit dem Thema Blockchain, da diese Technologie die M{\"o}glichkeit bietet Dokumente f{\"a}lschungssicher, irreversible und dezentral zu speichern.\footnote{\cite[S.~312]{Neugebauer.2018}} Satoshi Nakamoto ver{\"o}ffentlichte 2008 ein Artikel mit dem Titel {\glqq}Bitcoin: A Peer-to-Peer Electronic Cash System{\grqq}, indem eine W{\"a}hrung ohne zentrale Entit{\"a}t erl{\"a}utert wurde.\footnote{\cite[S.~1]{SatoshiNakamoto.}} Diese Ausarbeitung stellt die Basis jeder heutigen Blockchain. Eine Blockchain ist eine auf Kryptographie basierende verkette Liste von Blöcken. In den Blöcken werden die Daten der Blockchain gespeichert. In den meisten Fällen handelt es sich hierbei um Überweisungen der Kryptowährung von einer zur anderen Person, jedoch besteht auch die Möglichkeit andere Daten in diesen Blöcken zu speichern.

Diese Ausarbeitung verfolgt zwei Hauptziele. Zunächst soll das theoretische Grundwissen der Blockchain-Technologie als Datenbank erläutert werden. Hierbei steht die strukturelle Aufbau und die Konsensbildung der Blöcke im Vordergrund. Aufbauend auf diesem Wissen verfolgt diese Ausarbeitung das Ziel eine Blockchain mit der Software Multichain\footnote{https://www.multichain.com/} zu erstellen und anschließend Transaktionen durchzuführen. Hierbei steht die Installation, Konfiguration und die Verwendung der Blockchain im Vordergrund.

In Kapitel \ref{sec:Theorie} wird das theoretische Fundament gebildet. Es erläutert den Transaktionsablauf, den Blockaufbau und die Konsensbildung der Blockchain. Aufbauend auf der Theorie folgt in Kapitel \ref{sec:Praxis} die praktische Anwendung der Blockchainsoftware Multichain. Hierbei wird erläutert wie die Software installiert, konfiguriert und anschließend verwendet wird. Die Ausarbeitung schließt mit einer kritischen Betrachtung in Kapitel \ref{sec:Kritik} und einem Fazit in Kapitel \ref{sec:Fazit} ab.