% !TEX root = ../document.tex
\section{Kritische Reflexion}
\label{sec:Kritik}
Es muss beachtet werden, dass das Thema Blockchain erst 2009 aufgekommen ist und die Benutzung der Blockchain, außerhalb der Kryptowährungen, nicht weit verbreitet ist. Demnach wurden noch nicht viele Projekte beziehungsweise Studien zu diesem Thema veröffentlicht. Außerdem ist auch zu erwähnen, dass die meiste Literatur sich mit dem Handeln von Kryptowährungen befasst und nicht auf die technische Theorie des Themas eingeht. Die Hauptquelle dieser Ausarbeitung ist über ein Synonym veröffentlicht worden und keiner kann die Authentizität der Ausarbeitung validieren. Außerdem ist zu erwähnen das diese Ausarbeitung auf einer Literaturrecherche basiert und somit nur ein Abbild der aktuellen Literatur darstellt.

Für die praktische Anwendung wurde ausschließlich eine Blockchainsoftware verwendet. Es wurde nicht in Betracht gezogen, dass andere Blockchains ganz anders funktionieren könnten. Es sollten in Zukunft weitere Anwendung betrachtet werden.

Außerdem stellte sich das ausgewählte Thema als fehlerhaft definiert dar. Der Titel \glqq{}Blockchain als verteiltes Datenbankmanagementsystem\grqq{} ist nicht richtig gewählt, da eine Blockchain eher als einzelne Tabelle zu definieren ist. Ein Datenbankmanagementsystem besitzt jedoch viel mehr Attribute, die in einer Blockchain nicht enthalten sind. Beispielsweise kann mit einem Datenbankmanagementsystem komplexe Manipulationsbefehle ausgeübt werden, jedoch ist die Blockchain nur in der Lage Datensätze an die Kette anzuhängen. Demnach wurde der Titel der Ausarbeitung in \glqq{}Blockchain als verteilte Datenbank\grqq{} umbenannt.