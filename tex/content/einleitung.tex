% !TEX root = ../document.tex
\section{Einleitung}
\label{sec:Einleitung}
Seit der Sch{\"o}pfung der ersten Bitcoins 2009 w{\"a}chst die Popularit{\"a}t der Kryptow{\"a}hrungen und der Blockchain-Technologie stetig an.\footnote{\cite[S.~312]{Neugebauer.2018}} Besonders die Industrie besch{\"a}ftigt sich intensiv mit dem Thema Blockchain, da diese Technologie die M{\"o}glichkeit bietet Dokumente f{\"a}lschungssicher, irreversible und dezentral zu speichern.\footnote{\cite[S.~312]{Neugebauer.2018}} Satoshi Nakamoto ver{\"o}ffentlichte 2008 ein Artikel mit dem Titel {\glqq}Bitcoin: A Peer-to-Peer Electronic Cash System{\grqq}, indem eine W{\"a}hrung ohne zentrale Entit{\"a}t erl{\"a}utert wurde.\footnote{\cite[S.~1]{SatoshiNakamoto.}} Dieser Artikel von Satoshi Nakamoto stellt die Basis jeder Blockchain.

Innerhalb dieser Ausarbeitung sollen zwei Ziele verfolgt werden. Zunächst soll das theoretische Grundwissen der Blockchain-Technologie als Datenbank erläutert werden. Aufbauend auf diesem Wissen verfolgt diese Ausarbeitung das Ziel eine Blockchain mit der Software Multichain\footnote{https://www.multichain.com/} zu erstellen und anschließend Transaktionen durchzuführen.

Das nachfolgende Kapitel erläutert die theoretischen Hintergründe der Blockchain-Technologie. In Kapitel \ref{sec:Praxis} wird eine Blockchain mit der Software Multichain erstellt und anschließend erläutert wie Transaktionen durchgeführt werden können. In Kapitel \ref{sec:Kritik} folgt eine kritische Reflexion dieser Ausarbeitung. Abschließend wird ein Fazit der Ausarbeitung gebildet.